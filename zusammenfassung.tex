\documentclass[10pt,a4paper]{scrartcl}

\usepackage[T1]{fontenc}
\usepackage[ngerman]{babel}
\usepackage[utf8]{inputenc}

\usepackage[margin=2.5cm]{geometry}

\title{Wahrscheinlichkeitsrechnung und Statistik}
\author{Christoph H\"usler $\langle$chuesler@hsr.ch$\rangle$ } 

\begin{document}
\maketitle
\section{Kombinatorik}
\begin{description}
\item[Produktregel] Auswahl von k aus n Objekten, die einzelnen Elemente sind \emph{unabhängig} voneinander: $$n^k$$
\item[Permutation] Mögliche Anordnungen (Reihenfolge, Permutation) von n Objekten $$n!$$
\item[Kombination] Auswahl von k aus n Objekten, wobei jedes Objekt nur einmal gewählt werden kann:
    $$\frac{n(n-1)\cdots(n-k+1)}{1 \cdot 2 \cdots k} = \frac{n!}{k!(n-k)!} = {n \choose k}$$
    Die erste dieser Formeln ist gut geeignet zur Umsetzung in Integer, da Divisionen immer aufgehen und Zwischenresultate ``klein'' bleiben.
\end{description}

Für kompliziertere Situationen, z.B. mit Nebenbedingungen, lässt sich oft Symmetrie ausnutzen.

\subsection{Erzeugende Funktion}
Schwierige kombinatorische Fragestellungen lassen sich oft in eine algebraisches oder analytisches Problem umformulieren, welches wir per Computer lösen können.

\paragraph{Beispiel} Bildung eines Betrags aus 1 und 5 Fr. Münzen.
\paragraph{Teilaufgabe 1} Es gibt 1 Variante, den Betrag mit 1 Fr. Münzen zu bilden.
Die erzeugende Funktion sieht wie folgt aus (geometrische Reihe): $$1 + 1x + 1x^2 + 1x^3 + 1x^4 + \dots = \frac{1}{1-x}$$ 

\paragraph{Teilaufgabe 2} Wenn der Betrag durch 5 teilbar ist, gibt es genau eine Art den Betrag mit 5 Fr. Münzen zu bilden, wenn nicht, keine.
$$1 + 0x + \dots + 1x^5 + 0x^6 + \dots + 1x^{10} + \dots = $$ $$1 + x^5 + x^{10} + x^{15} + \cdots = \frac{1}{1-x^5}$$

\paragraph{Interpretation}
Der Koeffizient $a$ von $ax^k$ sagt aus, auf wie viele Arten ein Betrag von k Fr.\ gebildet werden kann. Das Produkt der beiden Reihen kombiniert dann die 1 Fr.\ und 5 Fr. Varianten (analog erweiterbar für weitere Münzen):
$$taylor\left(\frac{1}{(1-x)(1-x^5)}, 0\right) = 1 + x + x^2 + x^3 + x^4 + 2x^5 + 2x^6 + \dots + 3x^{10} + \dots$$

Äquivalent kann das Problem auch als Serie von Vektoren ausgedrückt werden, welche per Faltung (Convolution) kombiniert werden.
\end{document} 
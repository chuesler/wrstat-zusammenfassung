\documentclass[10pt,a4paper]{scrartcl}

\usepackage[T1]{fontenc}
\usepackage[ngerman]{babel}
\usepackage[utf8]{inputenc}

\usepackage{amssymb}

\usepackage[margin=2.5cm]{geometry}

\renewcommand{\arraystretch}{1.2}

\title{Wahrscheinlichkeitsrechnung und Statistik}
\author{Christoph H\"usler $\langle$chuesler@hsr.ch$\rangle$ } 

\begin{document}
\maketitle
\section{Kombinatorik}
\begin{description}
\item[Produktregel] Auswahl von k aus n Objekten, die einzelnen Elemente sind \emph{unabhängig} voneinander: $$n^k$$
\item[Permutation] Mögliche Anordnungen (Reihenfolge, Permutation) von n Objekten $$n!$$
\item[Kombination] Auswahl von k aus n Objekten, wobei jedes Objekt nur einmal gewählt werden kann:
    $$\frac{n(n-1)\cdots(n-k+1)}{1 \cdot 2 \cdots k} = \frac{n!}{k!(n-k)!} = {n \choose k}$$
    Die erste dieser Formeln ist gut geeignet zur Umsetzung in Integer, da Divisionen immer aufgehen und Zwischenresultate ``klein'' bleiben.
\end{description}

Für kompliziertere Situationen, z.B. mit Nebenbedingungen, lässt sich oft Symmetrie ausnutzen.

\subsection{Erzeugende Funktion}
Schwierige kombinatorische Fragestellungen lassen sich oft in eine algebraisches oder analytisches Problem umformulieren, welches wir per Computer lösen können.

\paragraph{Beispiel} Bildung eines Betrags aus 1 und 5 Fr. Münzen.
\paragraph{Teilaufgabe 1} Es gibt 1 Variante, den Betrag mit 1 Fr. Münzen zu bilden.
Die erzeugende Funktion sieht wie folgt aus (geometrische Reihe): $$1 + 1x + 1x^2 + 1x^3 + 1x^4 + \dots = \frac{1}{1-x}$$ 

\paragraph{Teilaufgabe 2} Wenn der Betrag durch 5 teilbar ist, gibt es genau eine Art den Betrag mit 5 Fr. Münzen zu bilden, wenn nicht, keine.
$$1 + 0x + \dots + 1x^5 + 0x^6 + \dots + 1x^{10} + \dots = $$ $$1 + x^5 + x^{10} + x^{15} + \cdots = \frac{1}{1-x^5}$$

\paragraph{Interpretation}
Der Koeffizient $a$ von $ax^k$ sagt aus, auf wie viele Arten ein Betrag von k Fr.\ gebildet werden kann. Das Produkt der beiden Reihen kombiniert dann die 1 Fr.\ und 5 Fr. Varianten (analog erweiterbar für weitere Münzen):
$$taylor\left(\frac{1}{(1-x)(1-x^5)}, 0\right) = 1 + x + x^2 + x^3 + x^4 + 2x^5 + 2x^6 + \dots + 3x^{10} + \dots$$

Äquivalent kann das Problem auch als Serie von Vektoren ausgedrückt werden, welche per Faltung (Convolution) kombiniert werden.

\section{Ereignisse und ihre Wahrscheinlichkeit}
\subsection{Ereignis}
Ein Ereignis ist immer verbunden mit einem (wiederholbaren) Experiment. Entscheidend ist der Versuchsausgang.

$$\Omega = \mbox{Menge der Elementar-Ereignisse} = \left\{ \mbox{alle möglichen Versuchsausgäng}e \right\}$$
Ein Ereignis ist eine Teilmenge von $\Omega$. Die Menge aller Ereignisse ist also die Menge aller Teilmengen von $\Omega$ (auch wenn viele davon unmöglich sind). \\
Beispiel: 7er-Würfel: $\Omega = \left\{ 1, 2, 3, 4, 5, 6, 7 \right\}$, 
2 6er-Würfel: $\Omega = \left[1, 2, 3, 4, 5, 6\right] \times \left[1, 2, 3, 4 ,5, 6\right]$
\\ \linebreak
Es können auch kompliziertere Ereignisse $A \subset \Omega$ definiert werden.
$$G = \mbox{``gerade Zahl gewúrfelt''} = \{2, 4, 6\}$$
$$U = \mbox{``ungerade Zahl gewúrfelt''} = \{1, 3, 5, 7\}$$
$$P = \mbox{``Primzahl gewúrfelt''} = \{2, 3, 5, 7\}$$

\subsection{Wahrscheinlichkeit}
Wir brauchen eine ``Übersetzungstabelle'' von der Alltagssprache in die Mengensprache. 

\begin{center}
\begin{tabular}{cc}
Ereignis A eingetreten & Versuchsausgang $\omega \in A$ \\ 
Ereignis A ist unmöglich & $\forall \omega: \omega \notin A \ \Rightarrow \ A = \varnothing \ \mbox{(das unmögliche Ereignis)}$  \\ 
Ereignis A trifft sicher ein & $\forall \omega: \omega \in A \ \Rightarrow \ A = \Omega \ \mbox{(das sichere Ereignis)}$ \\ 
A und B & $A \cap B$ \\ 
A oder B & $A \cup B$ \\ 
nicht A & $\Omega \setminus A$ \\ 
wenn A dann B & $A \subset B$  \\ 
A unter der Bedingung B & $A\ |\ B$ \\
\end{tabular}
\end{center}

Die Wahrscheinlichkeit eines Ereignisses $A$ wird geschrieben als $P(A)$.

$$P(A) = \lim_{\mbox{\footnotesize\#Versuche} \rightarrow \infty} \frac{\mbox{Anzahl Eintreten von A}}{\mbox{Anzahl Versuche}}$$

Beispiel fairer (7er-)Würfel: $$P(\{7\}) = \frac{1}{7},\ P(G) = \frac{3}{7}, \ P(\varnothing) = 0, \ P(\Omega) = 1$$

Die gemessenen Werte konvergieren mit einem Fehler der Grössenordnung $\frac{1}{\sqrt{n}}$ (eine Nachkommastelle mehr entspricht $10^2$ mal mehr Experimenten).

\subsubsection{Formeln für Wahrscheinlichkeiten} 

\begin{center}
\begin{tabular}{cc}
$A \subset B$ & $P(A) \leq P(B)$ \\
$\varnothing \subset A \subset \Omega$ & $0 = P(\varnothing) \leq P(A) \leq P(\Omega) = 1$ \\
$\bar{A}$ & $P(\bar{A}) + P(A) = 1 \Rightarrow P(\bar{A}) = P(\Omega \setminus A) = 1 - P(A)$ \\
$A \cup B$ & $ P(A\cup B) = P(A) + P(B) - P(A\cap B)$ \\
$A \cap B$ & Wenn A und B unabhängig: $P(A)P(B)$, sonst keine Formel \\
A, B unabhängig & $P(A\cap B) = P(A)P(B)$\\[2pt]
$A\ |\ B$ & $P(A\ |\ B) = \frac{P(A\cap B)}{P(B)}$ \\[2pt]
$A\ |\ B$ wenn $A\ |\ B_i$ vorliegen & $ P(A\ |\ B) = \sum_{i=1}^n \frac{P(A\ |\ B_i)}{P(B_i)} P(B_i) $
\end{tabular}
\end{center}

Intuition für diese Regeln ist die ``Flächenmessung'' (manchmal tatsächlich der Fall, z.B. Dartspiel).

%I still can't compile tex to pdf :< That's why I haven't pushed any changes. Very sorry. Uncompiled and untested stuff follows...

\subsection{DNA-Test vor Gericht}
Mögliche Ereignisse:
\begin{itemize}
\item $D$ = DNA-Test liefert eine Übereinstimmung
\item $T$ = Angeklagter ist Täter
\end{itemize}

Anklage: $P(D|T)$ //"Wenn es der Täter ist, muss das Ergebnis mit seiner DNA übereinstimmen"
Gericht: $P(T|D)$ // Wie wahrscheinlich ist die Übereinstimmung, wenn der Test Ja sagt.
Firma: $P(D|T)$  // testet, ob DNA-Test bei Übereinstimmung Ja sagt.

Wir wollen eine Schlussumkehr. --> Frage: Wie kommt man von dem, was die Firma gemessen hat, zu dem Umgekehrten, wie kommt man von $P(D|T)$ zu $P(T|D)$?

$P(A \cap B)$ = $P(B|A)$*$P(A)$ = $P(A|B)$*$P(B)$

Schlussfolgerung:
$P(A|B)$ = $P(B|A)$ * \frac{$P(A)$}{$P(B)$}
$P(B|A)$ = $P(A|B)$ * \frac{$P(B)$}{$P(A)$}

= Satz von Bayes

Im Gerichtssaal: $P(T|D)$ = $P(D|T)$ * \frac{$P(T)$}{$P(D)$}
Für ein wasserdichtes Alibi gilt: $P(T)$ = 0 --> $P(T|D)$ = 0 // Wasserdichtes Alibi schlägt DNA-Test

Je grösser $P(D)$ (die Übereinstimmung von Samples bei Tests, z.B. bei Blutgruppen-Test), desto kleiner $P(T|D)$, also die Wahrscheinlichkeit, dass man von diesem Test auf den Täter schliessen kann.


\subsection{HIV-Test}
2 Ereignisse:
\begin{itemize}
\item $H$ = hat Krankheit
\item $T$ = Test zeigt HIV an
\end{itemize}
«b
$P(H)$ = 0.0001
$P(T|H)$ = 0.999 // Wahrscheinlichkeit, dass ein Test HIV bei einer infizierten Person anzeigt
$P(\bar{T}|\bar{H})$ = 0.9999 // Wahrscheinlichkeit, dass Test NICHT HIV anzeigt, wenn die Person NICHT infiziert ist.

aber $P(H|T)$ = $P(T|H)$ * \frac{$P(H)$}{$P(T)$}

$P(T)$ = $P(T|H)$*$P(H)$ + $P(T|\bar{H})$ * $P(\bar{H})$
	 = 0.999 * 0.0001 + (1-0.9999) * (1-0.0001)
	 = 0.00019989 = \approx 0.0002

$P(H|T)$ = 0.999 * \frac{0.0001}{0.0002} \approx 0.5


Umgekehrt:
$P(\bar{H}|\bar{T})$ = $P(\bar{T}|\bar{H})$*\frac{$P(\bar{H})$}{$P(\bar{T})$}
 = 0.9999 * \frac{0.9999}{0.9998} \approx 1

\subsection{Karies bei Kindern + Schokoladenkonsum}
%Prüfungsaufgabe
$Z$ = {schlechte Zähne}
$S$ = {Schoggikonsum}

Wir wissen:
\begin{itemize}
\item $P(Z)$ = 0.06
\item $P(S|Z)$ = \frac{2}{3}
\item $P(S|\bar{Z})$ = 0.17
\end{itemize}

Gesucht: $P(Z|S)$ = $P(S|Z)$ * \frac{$P(Z)$}{$P(S)$}
Noch unbekannt: $P(S)$ --> Totale Wahrscheinlichkeit $P(S|Z)$ * $P(Z)$ + $P(S|\bar{Z})$ * $P(\bar{Z})$
 = \frac{2}{3} * 0.06 + 0.17 (1-0.06) = 0.1998 \approx 20\%

 $P(Z|S)$ = \frac{2}{3} * \frac{0.06}{0.2} = 20\%


\subsection{Ziegen und Autos}
3 Türen: 1 Auto, 2 Ziegen
1 von 3 Türen wählen
2 Strategien: Bleiben, Wechseln

Ereignisse:
\begin{itemize}
\item $G$ = {Auto gewonnen}
\item $A$ = {erste Wahl war ein Auto} = \bar{Z}
\item $Z$ = {erste Wahl war eine Ziege}
\end{itemize}

Finde Gewinnwahrscheinlichkeit
$P(G)$ = P(G|A) * P(A) + P(G|Z) * P(Z)

Strategie \"Bleiben\": 1 * \frac{1}{3} + 0 * \frac{2}{3} = \frac{1}{3}
Strategie \"Wechseln\": 0 * \frac{1}{3} + 1 * \frac{2}{3} = \frac{2}{3}

--> \"Wechseln\" ist doppelt so gut wie \"Bleiben\".

\subsubsection{Strategie mit Münzwurf festgelegt}
$B$: Bleibstrategie wurde gewählt
$W$: Wechselstrategie wurde gewählt

$P(G)$ = $P(G|B)$*$P(B)$ + $P(G|W)$*$P(W)$
	= \frac{1}{3}*\frac{1}{2} + \frac{2}{3}*\frac{1}{2} = \frac{1}{2}

Optimale Strategie: Immer wechseln.

\end{document}
\documentclass[10pt,a4paper]{scrartcl}

\usepackage[T1]{fontenc}
\usepackage[ngerman]{babel}
\usepackage[utf8]{inputenc}

\usepackage{amssymb}

\usepackage[margin=2.5cm]{geometry}

\renewcommand{\arraystretch}{1.2}

\title{Wahrscheinlichkeitsrechnung und Statistik}
\author{Christoph H\"usler $\langle$chuesler@hsr.ch$\rangle$ } 

\begin{document}
\maketitle
\section{Kombinatorik}
\begin{description}
\item[Produktregel] Auswahl von k aus n Objekten, die einzelnen Elemente sind \emph{unabhängig} voneinander: $$n^k$$
\item[Permutation] Mögliche Anordnungen (Reihenfolge, Permutation) von n Objekten $$n!$$
\item[Kombination] Auswahl von k aus n Objekten, wobei jedes Objekt nur einmal gewählt werden kann:
    $$\frac{n(n-1)\cdots(n-k+1)}{1 \cdot 2 \cdots k} = \frac{n!}{k!(n-k)!} = {n \choose k}$$
    Die erste dieser Formeln ist gut geeignet zur Umsetzung in Integer, da Divisionen immer aufgehen und Zwischenresultate ``klein'' bleiben.
\end{description}

Für kompliziertere Situationen, z.B. mit Nebenbedingungen, lässt sich oft Symmetrie ausnutzen.

\subsection{Erzeugende Funktion}
Schwierige kombinatorische Fragestellungen lassen sich oft in eine algebraisches oder analytisches Problem umformulieren, welches wir per Computer lösen können.

\paragraph{Beispiel} Bildung eines Betrags aus 1 und 5 Fr. Münzen.
\paragraph{Teilaufgabe 1} Es gibt 1 Variante, den Betrag mit 1 Fr. Münzen zu bilden.
Die erzeugende Funktion sieht wie folgt aus (geometrische Reihe): $$1 + 1x + 1x^2 + 1x^3 + 1x^4 + \dots = \frac{1}{1-x}$$ 

\paragraph{Teilaufgabe 2} Wenn der Betrag durch 5 teilbar ist, gibt es genau eine Art den Betrag mit 5 Fr. Münzen zu bilden, wenn nicht, keine.
$$1 + 0x + \dots + 1x^5 + 0x^6 + \dots + 1x^{10} + \dots = $$ $$1 + x^5 + x^{10} + x^{15} + \cdots = \frac{1}{1-x^5}$$

\paragraph{Interpretation}
Der Koeffizient $a$ von $ax^k$ sagt aus, auf wie viele Arten ein Betrag von k Fr.\ gebildet werden kann. Das Produkt der beiden Reihen kombiniert dann die 1 Fr.\ und 5 Fr. Varianten (analog erweiterbar für weitere Münzen):
$$taylor\left(\frac{1}{(1-x)(1-x^5)}, 0\right) = 1 + x + x^2 + x^3 + x^4 + 2x^5 + 2x^6 + \dots + 3x^{10} + \dots$$

Äquivalent kann das Problem auch als Serie von Vektoren ausgedrückt werden, welche per Faltung (Convolution) kombiniert werden.

\section{Ereignisse und ihre Wahrscheinlichkeit}
\subsection{Ereignis}
Ein Ereignis ist immer verbunden mit einem (wiederholbaren) Experiment. Entscheidend ist der Versuchsausgang.

$$\Omega = \mbox{Menge der Elementar-Ereignisse} = \left\{ \mbox{alle möglichen Versuchsausgäng}e \right\}$$
Ein Ereignis ist eine Teilmenge von $\Omega$. Die Menge aller Ereignisse ist also die Menge aller Teilmengen von $\Omega$ (auch wenn viele davon unmöglich sind). \\
Beispiel: 7er-Würfel: $\Omega = \left\{ 1, 2, 3, 4, 5, 6, 7 \right\}$, 
2 6er-Würfel: $\Omega = \left[1, 2, 3, 4, 5, 6\right] \times \left[1, 2, 3, 4 ,5, 6\right]$
\\ \linebreak
Es können auch kompliziertere Ereignisse $A \subset \Omega$ definiert werden.
$$G = \mbox{``gerade Zahl gewúrfelt''} = \{2, 4, 6\}$$
$$U = \mbox{``ungerade Zahl gewúrfelt''} = \{1, 3, 5, 7\}$$
$$P = \mbox{``Primzahl gewúrfelt''} = \{2, 3, 5, 7\}$$

\subsection{Wahrscheinlichkeit}
Wir brauchen eine ``Übersetzungstabelle'' von der Alltagssprache in die Mengensprache. 

\begin{center}
\begin{tabular}{cc}
Ereignis A eingetreten & Versuchsausgang $\omega \in A$ \\ 
Ereignis A ist unmöglich & $\forall \omega: \omega \notin A \ \Rightarrow \ A = \varnothing \ \mbox{(das unmögliche Ereignis)}$  \\ 
Ereignis A trifft sicher ein & $\forall \omega: \omega \in A \ \Rightarrow \ A = \Omega \ \mbox{(das sichere Ereignis)}$ \\ 
A und B & $A \cap B$ \\ 
A oder B & $A \cup B$ \\ 
nicht A & $\Omega \setminus A$ \\ 
wenn A dann B & $A \subset B$  \\ 
A unter der Bedingung B & $A\ |\ B$ \\
\end{tabular}
\end{center}

Die Wahrscheinlichkeit eines Ereignisses $A$ wird geschrieben als $P(A)$.

$$P(A) = \lim_{\mbox{\footnotesize\#Versuche} \rightarrow \infty} \frac{\mbox{Anzahl Eintreten von A}}{\mbox{Anzahl Versuche}}$$

Beispiel fairer (7er-)Würfel: $$P(\{7\}) = \frac{1}{7},\ P(G) = \frac{3}{7}, \ P(\varnothing) = 0, \ P(\Omega) = 1$$

Die gemessenen Werte konvergieren mit einem Fehler der Grössenordnung $\frac{1}{\sqrt{n}}$ (eine Nachkommastelle mehr entspricht $10^2$ mal mehr Experimenten).

\subsubsection{Formeln für Wahrscheinlichkeiten} 

\begin{center}
\begin{tabular}{cc}
$A \subset B$ & $P(A) \leq P(B)$ \\
$\varnothing \subset A \subset \Omega$ & $0 = P(\varnothing) \leq P(A) \leq P(\Omega) = 1$ \\
$\bar{A}$ & $P(\bar{A}) + P(A) = 1 \Rightarrow P(\bar{A}) = P(\Omega \setminus A) = 1 - P(A)$ \\
$A \cup B$ & $ P(A\cup B) = P(A) + P(B) - P(A\cap B)$ \\
$A \cap B$ & Wenn A und B unabhängig: $P(A)P(B)$, sonst keine Formel \\
A, B unabhängig & $P(A\cap B) = P(A)P(B)$\\[2pt]
$A\ |\ B$ & $P(A\ |\ B) = \frac{P(A\cap B)}{P(B)}$ \\[2pt]
$A\ |\ B$ wenn $A\ |\ B_i$ vorliegen & $ P(A\ |\ B) = \sum_{i=1}^n \frac{P(A\ |\ B_i)}{P(B_i)} P(B_i) $
\end{tabular}
\end{center}

Intuition für diese Regeln ist die ``Flächenmessung'' (manchmal tatsächlich der Fall, z.B. Dartspiel).

\end{document} 